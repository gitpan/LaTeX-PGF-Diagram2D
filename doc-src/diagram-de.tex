\documentclass[ngerman,origlongtable]{scrartcl}
\KOMAoptions{paper=a4}
\KOMAoptions{fontsize=12pt}
\usepackage{ifpdf}
\usepackage[utf8]{inputenc}
\usepackage[T1]{fontenc}
\usepackage{textcomp}
\usepackage[intlimits]{amsmath}
\usepackage{amssymb}
% \usepackage{amsfonts,amssymb,amscd}
% \usepackage{trfsigns}
\usepackage{mathptmx}
\usepackage[scaled=.92]{helvet}
\usepackage{courier}
\usepackage{array}
\usepackage{enumerate}
\usepackage{longtable}
\usepackage{latexsym}
\usepackage{varioref}
\usepackage{makeidx}
\usepackage{listings}
\lstset{numbers=left, numberstyle=\tiny, numbersep=9pt, basicstyle=\small}
\lstset{frame=single,frameround=tttt}
\usepackage{color}
\ifpdf
\usepackage[activate=normal]{pdfcprot}
\usepackage[pdftex]{graphicx}
% \usepackage{epstopdf}
\pdfcompresslevel=9
\usepackage[
  pdftex,
  a4paper=true,
  colorlinks=true,
  bookmarks,
  bookmarksnumbered,
  linkcolor=black,
  pdfpagemode=UseNone,
  pdfstartview={XYZ 0 835 1.25}
]{hyperref}
\pdfpageattr {/Group << /S /Transparency /I true /CS /DeviceRGB>>}
\else
\usepackage[dvips]{graphicx}
\DeclareGraphicsRule{.png}{eps}{.bb}{`bmeps #1}
\usepackage[dvips]{hyperref}
\fi


\setlength{\parindent}{0cm}


\renewcommand*{\sectfont}{\bfseries}

\newcommand{\startofdocument}[0]{%
\maketitle%
\cleardoublepage%
\ifpdf\pdfbookmark[1]{\contentsname}{tocanc}\fi%
\tableofcontents%
\cleardoublepage%
}

\DeclareMathOperator{\euler}{\text{e}}
\DeclareMathOperator{\oplaplace}{\triangle}
\DeclareMathOperator{\opLaplace}{{\mathfrak L}}
\DeclareMathOperator{\iopLaplace}{{\opLaplace}^{-1}}
\DeclareMathOperator{\definedas}{\overset{=}{\text{\tiny{Def}}}}
\DeclareMathOperator{\definedarrow}{\overset{\longleftrightarrow}{\text{\tiny{Def}}}}
\DeclareMathOperator{\lcorrelates}{\laplace}
\DeclareMathOperator{\ilcorrelates}{\Laplace}
\DeclareMathOperator{\mustbe}{\overset{\text{\tiny !}}{=}}
\DeclareMathOperator{\coj}{\text{j}}
\DeclareMathOperator{\element}{\in}
\DeclareMathOperator{\notelement}{!\in}
\newcommand{\ora}[1]{\overrightarrow{#1}}
\newcommand{\co}[1]{\underline{#1}}
\newcommand{\cove}[1]{\overrightarrow{\underline{#1}}}
\DeclareMathOperator{\grad}{grad}
\DeclareMathOperator{\divergence}{div}
\DeclareMathOperator{\rot}{rot}
\newcommand{\atpos}[2]{{\left.#1\right|}_{#2}}
\newcommand{\inrange}[3]{{\left.#1\right|}_{#2}^{#3}}
\newcommand{\intd}[0]{\mathrm{d}}
\DeclareMathOperator{\arcsinh}{arcsinh}
\DeclareMathOperator{\card}{card}
\DeclareMathOperator{\Res}{Res}
\DeclareMathOperator{\expe}{e}

\newcommand{\begriff}[2]{%
  \emph{#1}\index{#2}%
}

\usepackage{attachfile}

\makeindex




% Image base file name
% Caption
\newcommand{\figimage}[2]{%
\begin{figure}%
{\centering%
\input{#1.tex}%
\caption{#2}%
\label{fig:#1}%
}%
\end{figure}%
}

% Linewidth factor
% Image base file name
% Caption
\newcommand{\figimagelw}[3]{%
\begin{figure}%
{\centering%
\resizebox{#1\linewidth}{!}{\input{#2.tex}}%
\caption{#3}%
\label{fig:#1}%
}%
\end{figure}%
}

% Width in cm
% Image base file name
% Caption
\newcommand{\figimagecm}[3]{}

% Image base file name
% Caption
\newcommand{\pngimage}[2]{%
\begin{figure}%
{\centering%
\ifpdf%
\input{#1.pdf}%
\else%
\input{#1}%
\fi%
\caption{#2}%
\label{fig:#1}%
}%
\end{figure}%
}

% Linewidth factor
% Image base file name
% Caption
\newcommand{\pngimagelw}[3]{%
\begin{figure}%
{\centering%
\ifpdf%
\includegraphics[width=#1\linewidth]{#2.pdf}%
\else%
\includegraphics[width=#1\linewidth]{#2}%
\fi%
\caption{#3}%
\label{fig:#1}%
}%
\end{figure}%
}

% Width in cm
% Image base file name
% Caption
\newcommand{\pngimagecm}[3]{%
\begin{figure}%
{\centering%
\ifpdf%
\includegraphics[width=#1cm]{#2.pdf}%
\else%
\includegraphics[width=#1cm]{#2}%
\fi%
\caption{#3}%
\label{fig:#1}%
}%
\end{figure}%
}

% Image base file name
% Caption
\newcommand{\mpimage}[2]{%
\begin{figure}%
{\centering%
\includegraphics{#1.mps}%
\caption{#2}%
\label{fig:#1}%
}%
\end{figure}%
}

% Linewidth factor
% Image base file name
% Caption
\newcommand{\mpimagelw}[3]{%
\begin{figure}%
{\centering%
\includegraphics[width=#1\linewidth]{#2.mps}%
\caption{#3}%
\label{fig:#1}%
}%
\end{figure}%
}

% Width in cm
% Image base file name
% Caption
\newcommand{\mpimagecm}[3]{%
\begin{figure}%
{\centering%
\includegraphics[width=#1cm]{#2.mps}%
\caption{#3}%
\label{fig:#1}%
}%
\end{figure}%
}


\newcommand{\gpimage}[2]{\mpimage{#1}{#2}}
\newcommand{\gpimagelw}[3]{\mpimagelw{#1}{#2}{#3}}
\newcommand{\gpimagecm}[3]{\mpimagecm{#1}{#2}{#3}}


\hyphenation{
Ar-beits-grup-pe
Win-dow-Event
Ar-beits-platz
Komp-ri-mier-ung
komp-ri-mier-te
Aus-gangs-ver-zeich-nis
Standard-ein-gabe
Dia-gramm
Dia-gram-Rand
Ska-len-teil-ung
}

\usepackage{fancybox}
\usepackage{pgf}
\usepackage{ngerman}
\newcommand{\version}[0]{1.00}
\author{Dipl.-Ing.~D.~Krause}
\title{\resizebox{10cm}{!}{LaTeX::PGF::Diagram2D}}
\ifpdf
\hypersetup{
  pdftitle={LaTeX::PGF::Diagram2D 1.00},
  pdfsubject={LaTeX::PGF::Diagram2D},
  pdfauthor={Dipl.-Ing.~D.~Krause}
}
\fi

\newcommand{\includepgfimage}[2]{%
\begin{figure}%
{\centering%
\input{#1.pgf}\caption{#2}\label{fig:#1}%
}%
\end{figure}%
}
\begin{document}
\startofdocument
\section{Einleitung}
Das Perl-Modul LaTeX::PGF::Diagram2D zeichnet 2D-Diagramme entsprechend dem
Standard \textsc{din}\,461 für die Verwendung mit \LaTeX{}.
Unterstützt werden nur solche Diagramme, aus denen
Werte ausgelesen werden können (in der \textsc{din} als Arbeitsdiagramme
bezeichnet).

Es können sowohl Funktionen als auch Messwerte geplottet werden
sowie parametrische Plots.

Das Modul erzeugt Diagramme als PGF-Dateien. Das \LaTeX{}-Paket
"`pgf"' erlaubt es, Graphiken im \LaTeX-Quelltext zu beschreiben,
diese Graphiken sind sowohl mit latex+dvips als auch mit pdflatex nutzbar.

Mit Code wie
\begin{lstlisting}
\begin{figure}%
{\centering%
\input{xxx.pgf}\caption{Meine Bildunterschrift}\label{blabla}%
}%
\end{figure}%
\end{lstlisting}
kann eine PGF-Datei im \LaTeX-Quelltext eingebunden werden.

Die Größe der Diagrammfläche (nicht wie bei anderen Plotprogrammen
die Größe des sich ergebenden gesamten Bildes) wird in cm vorgegeben.
Beim Plotten von Funktionen als Kurve werden standardmäßig
Bezier-Spline-Segmente
verwendet. Linienzüge (polylines) werden nur verwendet, wenn dies explizit
angefordert wird.\\
Das Ziel des Modules besteht nicht darin, andere Programme wie
GnuPlot\ldots{} zu ersetzen, vielmehr erfordert ein sinnvoller Einsatz
des Modules die Kombination mit anderer Mathematiksoftware (im Tutorial
werden GnuPlot\footnote{\url{http://gnuplot.sourceforge.net}}
und wxMaxima\footnote{\url{http://maxima.sourceforge.net}}
benutzt).


\clearpage
\section{Tutorial}
\subsection[Funktion]{Beispiel 1: Funktionsverlauf plotten}
\figimage{diagf001}{Beispielschaltung}
Es soll eine Schaltung entsprechend Abb.\vref{fig:diagf001} mit
\begin{align*}U_{\text{q}}&=1\,\text{V}&R_{\text{i}}&=4\,\Omega\end{align*}
untersucht werden. An eine Spannungsquelle mit der Quellspannung
1\,V und einem Innenwiderstand von 4\,\(\Omega\) wird ein Lastwiderstand
\(R_{\text{L}}\) angeschlossen.

Gesucht wird der Strom \(I\) in Abhängigkeit von \(R_{\text{L}}\) für
\(0\leq R_{\text{L}}\leq 10\,\Omega\).

Wie man leicht sieht, gilt:
\begin{align*}I&=\frac{U_{\text{q}}}{R_{\text{L}}+R_{\text{i}}}\end{align*}

Mit GnuPlot verschaffen wir uns zunächst eine Übersicht über den Verlauf:
\begin{lstlisting}[language=GnuPlot]
Uq=1.0
Ri=4.0
set xrange [0:10.0]
plot Uq/(Ri+x) title "Strom"
\end{lstlisting}
Wir sehen einen maximalen Kurzschlussstrom von 0,25\,A, die Kurve fällt
asymptotisch ab. Für die Darstellung auf Papier entscheiden wir uns für
eine Breite von 10\,cm und eine Höhe von 6\,cm, wir wählen den Wertebereich
bis 0,3\,A.
\clearpage
\lstinputlisting[float,title={../examples/test001a.pl},language=Perl]{../examples/test001a.pl}
Mit der Datei ../examples/test001a.pl fertigen wir den ersten Plot an.

Das Modul LaTeX::PGF::Diagram2D wird mit einer "`use"'-Anweisung eingebunden
(Zeile 3).

Die Funktion \textit{I}, die als Argument den Lastwiderstands-Wert
erwartet, wird in den Zeilen 8\ldots{}13 definiert.

Mit "`LaTeX::PGF::Diagram2D->new"' wird ein neues Diagramm angelegt, hier mit
einer Breite von 10\,cm und einer Höhe von 6\,cm (Zeile 16).

Das Diagramm wird in ein \LaTeX{}-Dokument eingebunden, das mit einer
Schriftgröße von 12\,pt erstellt wird. Dies wird in Zeile 18
konfiguriert.

Die untere \(x\)-Achse (b: bottom)
soll einen Wertebereich von 0,0\ldots{}10,0 abdecken
und linear geteilt sein. Gitternetzlinien für die \(x\)-Achse werden
jeweils im Abstand von 1 gezeichnet. Skalenbeschriftungen werden jeweils
im Abstand 1 angebracht (Zeilen 20 und 21).

Die linke \(y\)-Achse (l: left)
soll einen Wertebereich von 0,0\ldots{}0,3 abdecken.
Gitternetzlinien sollen jeweils im Abstand 0,05 gezeichnet werden,
Skalenbeschriftungen werden jeweils im Abstand 0,1 angebracht (also
an jeder zweiten Gitternetzlinie) (Zeilen 22 und 23).

Mit der Diagramm-Methode \textit{axis()\/} wird eine Referenz auf ein
Achsen-Objekt geholt. Für das Objekt werden nacheinander verschiedene
Methoden zum Setup aufgerufen. Bei jeder dieser Methoden ist der
Rückgabewert wieder die originale Objekt-Referenz, damit sind die
Methodenaufrufe kaskadierbar.

Mit der Methode \textit{plot()\/} des Diagrammes erhalten wir ein
Plot-Objekt (Zeile 23). Dieses Objekt wird so konfiguriert, dass es
die Funktion \textit{I()\/} benutzt, um Werte zu ermitteln (Zeile 26).
Hierzu wird die Funktion
\textit{set\textunderscore{}xy\textunderscore{}fct()\/} aufgerufen,
als Argument erhält sie eine Referenz auf die Funktion.

Das Diagramm wird als Datei "`../examples/test001a.pgf"' gespeichert
(Zeile 28).

\includepgfimage{../examples/test001a}{Diagramm für den Stromverlauf}
\clearpage
\subsection[Beschriftung]{Beispiel 2: Achsenbeschriftung und Einheit}
\lstinputlisting[float,language=Perl,title={../examples/test002a.pl}]{../examples/test002a.pl}
\includepgfimage{../examples/test002a}{Diagramm mit Achsenbeschriftung und Einheiten}
Mit der Methode \textit{set\textunderscore{}label()\/} wird die
Achsenbeschriftung eingestellt. Mit
\textit{set\textunderscore{}unit()\/} wird die Einheit angegeben.
Die Angabe der Größe erfolgt im Mathematikmodus,
die Einheiten hingegen werden aufrecht stehend gesetzt. Der verwendete
Text wird an \LaTeX{} zur Verarbeitung übergeben, es kann also das gleiche
Markup verwendet werden, wie im normalen \LaTeX{}-Quelltext.
Da der Backslash auch in Perl eine Sonderbedeutung hat, müssen für jeden
Backslash, den \LaTeX{} sehen soll zwei Backslashes im Perl-Quelltext
stehen.

Die Achsenbeschriftungen werden automatisch mit den Pfeilen in Richtung
wachsender Werte versehen, die Einheit wird zwischen den vorletzten
und den letzten Skalenwert geschrieben.
\clearpage
\subsection[Platz für Einheit]{Beispiel 3: Mehr Platz für die Einheit}
\includepgfimage{../examples/test003a}{Mehr Platz um Einheit der \(x\)-Achse}
Im vorangegangenen Beispiel stehen die Texte "`9"', "`\(\Omega\)"' und "`10"'
sehr eng gedrängt. Es ist gestattet, eine oder mehrere Skalenwerte wegzulassen,
um Platz für die Einheit zu schaffen. Dabei darf aber nicht die Beschriftung
"`0"' weggelassen werden.

Mit der Methode \textit{set\textunderscore{}omit()\/} wird festgelegt,
wieviele Skalenwerte an der jeweiligen Achse weggelassen werden sollen.
\lstinputlisting[float,language=Perl,title={../examples/test003a.pl}]{../examples/test003a.pl}
\clearpage
\subsection[Mehrere Kurven]{Beispiel 4: Mehrere Kurven}
\includepgfimage{../examples/test004a}{Mehrere Kurven in einem Diagramm}
Es können auch mehrere Funktionen in einem Diagramm geplottet werden,
hierzu werden einfach mehrere Plot-Objekte angelegt und konfiguriert.

Im Beispiel wird ein zweiter Plot vorgenommen, für den der Innenwiderstand
der Spannungsquelle auf 6\,\(\Omega\) erhöht wird.

Werden Variablen geändert, die als Parameter in die genutzten Funktionen
eingehen, muss vorher die Methode \textit{finish()\/} für die bereits
bestehenden Plot-Objekte aufgerufen werden. Diese Methode berechnet die
für den Plot zu verwendenden Funktionswerte.
Die Methode \textit{finish()\/} darf erst aufgerufen werden, wenn die
Konfiguration des Plot-Objektes abgeschlossen ist.

Im Beispiel wird in Zeile 28 die Berechnung der Funktionswerte für den
ersten Plot mit \(R_{\text{i}}=4\,\Omega\) veranlasst.
Anschließend wird die Variable "`\textdollar{}Ri"' auf 6.0 geändert.
Danach wird ein zweites Plot-Objekt mit derselben Funktion angelegt.
Hier kommt der geänderte \(R_{\text{i}}\)-Wert zum Tragen
(der \textdollar{}Ri-Wert zu dem Zeitpunkt, an dem die Methode
\textit{finish()\/} des Plot-Objektes aufgerufen wird).

\lstinputlisting[float,language=Perl,title={../examples/test004a.pl}]{../examples/test004a.pl}
\clearpage
\subsection[Text-Labels]{Beispiel 5: Text-Labels}
\includepgfimage{../examples/test005a}{Diagramm mit Text-Labeln}
Mit Hilfe von Text-Labeln können verschiedene Kurven auseinandergehalten
werden.
Die Diagramm-Methode \textit{label()\/} legt ein Text-Label an.
Als Argumente werden zunächst die Achsennamen angegeben, auf die sich
die Koordinaten beziehen, dann die Koordinaten, dann der Labeltext.
Optional können noch Angaben zum Text-Anker erfolgen.

Im Beispiel werden die Kommandos
"`\textbackslash{}colorbox"' und "`\textbackslash{}ovalbox"' aus dem
"`fancybox"'-Paket benutzt, um optisch ansprechende Labels zu erzeugen
(Zeilen 34 bis 39).
\lstinputlisting[float,language=Perl,title={../examples/test005a.pl}]{../examples/test005a.pl}
\clearpage
\subsection[Zwei Skalen]{Beispiel 6: Mehrere Skalen}
\includepgfimage{../examples/test006a}{Diagramm mit zwei Skalenteilungen}
Als zweite Kurve soll jetzt die am Lastwiderstand umgesetzte Leistung
in Abhängigkeit vom Lastwiderstand mit eingezeichnet werden.\\
Diese ergibt sich als
\begin{align*}P&=\frac{{U_{\text{q}}}^2R_{\text{L}}}{{(R_{\text{L}}+R_{\text{i}})}^2}\end{align*}
Mit GnuPlot verschaffen wir uns zunächst einen Überblick:
\begin{lstlisting}[language=GnuPlot]
Uq=1.0
Ri=4.0
set xrange [0:10.0]
plot Uq*Uq*x/((Ri+x)*(Ri+x)) title "Leistung"
\end{lstlisting}
Wir sehen, dass bei \(R_{\text{i}}=4\,\Omega\) ein Maximum
\(P_{\text{max}}=0,0625\,\text{W}\) auftritt.
Wir entscheiden uns dafür, die Diagramm-Höhe auf 8\,cm zu erhöhen. Die
linke \(y\)-Achse deckt jetzt den Bereich bis 0,4\,A ab, die rechte
\(y\)-Achse soll den Bereich bis 0,08\,W abdecken.

Weiterhin müssen für die rechte Achse einige Abstände festgelegt werden:
\begin{itemize}
\item	\textit{set\textunderscore{}tic\textunderscore{}offset()\/}\\
legt den Abstand zwischen dem Diagrammrand und der rechten Kante der
Skalenbeschriftungen fest. Der Standardwert von 2\,mm ist für die linke
Skale sinnvoll, für die rechte Skale muss die Einstellung an die jeweiligen
Skalenwerte angepasst werden.
\item	\textit{set\textunderscore{}label\textunderscore{}offset()\/}\\
legt den Abstand zwischen Diagrammrand und der Mitte der Achsenbeschriftung
fest.
\item	\textit{set\textunderscore{}border()\/}\\
legt den Abstand zwischen dem Diagrammrand und dem Bildrand fest.
\end{itemize}
\lstinputlisting[float,language=Perl,title={../examples/test006a.pl}]{../examples/test006a.pl}
\clearpage
\subsection[Einheiten]{Beispiel 7: Einheiten}
\includepgfimage{../examples/test007a}{Geänderte Einheiten}
Die Darstellung kann verbessert werden, indem als Einheiten mA bzw.
mW gewählt werden.

Hierzu müssen die Berechnungsfunktionen geändert werden (Ergebnis
vor der Rückgabe mit Faktor 1000 multiplizieren) und die Achseneinstellungen
angepasst werden.

Auch die Koordinaten der Text-Labels müssen angepasst werden.
\lstinputlisting[float,language=Perl,title={../examples/test007a.pl}]{../examples/test007a.pl}
\clearpage
\subsection[Ableitung angeben]{Beispiel 8: Angabe der ersten Ableitung}
\includepgfimage{../examples/test008a}{Diagramm mit Angabe der ersten Ableitungen}
Es ist günstig, beim Kurven-Plotten die erste Ableitung
\begin{align*}\frac{\intd Y}{\intd X}\end{align*}
der Funktion \(Y=f(X)\) mit anzugeben.

Kurven werden mittels Bezier-Splines realisiert, für die Berechnung der
Kontrollpunkte wird die erste Ableitung benötigt.
Ist keine Funktion zur Berechnung der Ableitung angegeben, so wird beim
Plot ein Natural Spline in die Funktionswerte gelegt und aus den Polynomen
der Spline-Segmente die erste Ableitung berechnet.

Für die Berechnung des Natural Splines muss ein lineares Gleichungssystem
gelöst werden, die Anzahl der Gleichungen ist das Vierfache der Anzahl der
Spline-Segmente. Das Lösen eines solchen Gleichungssystemes nimmt umso mehr
Rechenzeit in Anspruch, umso größer die Anzahl der Spline-Segmente ist.

Mit wxMaxima kann mit
\begin{lstlisting}
diff((1000*Uq/(Ri+RL)), RL);
ratsimp(diff((1000*Uq*Uq*RL/((Ri+RL)*(Ri+RL))), RL));
\end{lstlisting}
die Ableitung der beiden Funktionen 
\begin{align*}
I(R_{\text{L}})&=\frac{1000\cdot{}U_{\text{q}}}{R_{\text{i}}+R_{\text{L}}}&P(R_{\text{L}})&=\frac{1000\cdot{}U_{\text{q}}R_{\text{L}}^2}{{(R_{\text{i}}+R_{\text{L}})}^2}\\
\intertext{als}
\frac{\intd I}{\intd R_{\text{L}}}&=-\frac{1000\cdot{}U_{\text{q}}}{{(R_{\text{i}}+R_{\text{L}})}^2}&\frac{\intd P}{\intd R_{\text{L}}}&=-\frac{1000\cdot{} U_{\text{q}}^2(R_{\text{L}}-R_{\text{i}})}{R_{\text{}L}^3+3R_{\text{i}}R_{\text{L}}^2+3R_{\text{i}}^2R_{\text{L}}+3R_{\text{i}}^3}\end{align*}
einfach gefunden werden.

Die Referenz auf die Funktion zur Berechnung der ersten Ableitung wird als
optionales zweites Argument beim Aufruf von
\textit{set\textunderscore{}xy\textunderscore{}fct()\/} angegeben
(Zeilen 39 und 40).
\lstinputlisting[float,language=Perl,title={../examples/test008a.pl}]{../examples/test008a.pl}
\clearpage
\subsection[Parametrischer Plot]{Beispiel 9: Parametrischer Plot}
\includepgfimage{../examples/test009a}{Parametrischer Plot}
Im Beispiel wird die spiralförmige Kurve
\begin{align*}
\tilde{k}:&&x&=5\expe^{-t/2\pi}\cos{t}&y&=5\expe^{-t/2\pi}\sin{t}&0&\leq{}t\leq{}6\pi
\end{align*}
geplottet.
Es werden die Funktionen
\begin{align*}
x(t)&=5\cdot\expe^{-t/2\pi}\cos{t}&y(t)&=5\cdot\expe^{-t/2\pi}\sin{t}\\
\intertext{und die Ableitungen}
\frac{\intd x}{\intd t}&=-\frac{1}{2\pi}(10\pi\sin{t}+5\cos{t})\cdot\expe^{-t/2\pi}&\frac{\intd y}{\intd t}&=-\frac{1}{2\pi}(5\sin{t}-10\pi\cos{t})\expe^{-t/2\pi}\end{align*}
benutzt. Für ein gegebenes \(t\) wird der \(x\)- bzw. \(y\)-Wert mit der
Funktion \textit{xfct()\/} bzw. \textit{yfct()\/} berechnet.
Die Ableitungen \(\tfrac{\intd x}{\intd t}\) bzw. \(\tfrac{\intd y}{\intd t}\)
werden durch die Funktionen \textit{dxdt()\/} bzw. \textit{dydt()\/}
berechnet.

Die Funktion
\textit{set\textunderscore{}parametric\textunderscore{}fct()\/}
muss als Argumente den Startwert und den Endwert des laufenden Parameters
\(t\) erhalten sowie die Referenzen auf die Funktionen
\textit{xfct()\/} und \textit{yfct()\/}. Als optionale Argumente folgen
die Referenzen auf die Funktionen
\textit{dxdt()\/} und \textit{dydt()\/}. Auch bei parametrischen
Kurvenplots wird die Angabe der ersten Ableitungen empfohlen.
\lstinputlisting[float,language=Perl,title={../examples/test009a.pl}]{../examples/test009a.pl}
\clearpage
\subsection[Polylinie]{Beispiel 10: Kurve oder Polylinie}
\includepgfimage{../examples/test011a}{Funktionsdarstellung als Polylinie}
Funktionen mit steilen Anstiegen, Unstetigkeiten, Knicken und
anderweitig unsanften
Verläufen werden u.\,U. in Form einer Polylinie
besser und schneller dargestellt als mittels einer Kurve (Natural Spline).

Im Beispiel wird eine Rechteckspannung mit überlagerten Störungen
dargestellt.

Mit der Methode
\textit{set\textunderscore{}lines()\/} des Plot-Objektes wird eine Darstellung
als Polylinie veranlasst. Beim Zeichnen der Polylinie werden nur die
Punkte benötigt, keine Anstiege. Damit entfällt das Lösen eines
Gleichungssystemes, da kein Natural Spline benötigt wird. Im Gegenzug muss
aber mit
\textit{set\textunderscore{}intervals()\/} die Anzahl der Segmente erhöht und
an die Funktion angepasst werden.
\clearpage
\lstinputlisting[float,language=Perl,title={../examples/test011a.pl}]{../examples/test011a.pl}
\clearpage
\subsection[Logarithmisch]{Beispiel 11: Logarithmische Skalenteilung}
\includepgfimage{../examples/test012a}{Tiefpass-Übertragungsfunktion}
Im Beispiel wird die Tiefpass-Übertragungsfunktion
\begin{align*}
\frac{v_U}{\text{dB}}&=20\cdot\lg{\left|\frac{1}{1+\coj\frac{f}{f_{\text{G}}}}\right|}\\[0.2em]
&=-10\lg{\left({\left(\frac{f}{f_{\text{G}}}\right)}^2+1\right)}\end{align*}
dargestellt. Für die \(x\)-Achse wird eine logarithmische Skalenteilung
benutzt.
Die Substitution
\begin{align*}
X&=\frac{f}{f_{\text{G}}}\\
\intertext{ergibt}
Y&=-10\lg{\left(X^2+1\right)}\\[0.2em]
\frac{\intd Y}{\intd X}&=-\frac{20X}{\left(X^2+1\right)\ln{10}}
\end{align*}
Mit der Achsen-Methode
\textit{set\textunderscore{}logarithmic()\/}
wird eine logarithmische Skalenteilung für die Achse eingestellt, Anfangs-
und Endwert müssen positiv sein.\\
Der Wert für
\textit{set\textunderscore{}tic\textunderscore{}step()\/} und
\textit{set\textunderscore{}grid\textunderscore{}step()\/} gibt hier nicht
einen Abstand sondern einen Faktor an.

Im Beispiel werden Gitternetzlinien jeweils für die Verzehnfachung des
\(x\)-Wertes gezogen.
\clearpage
\lstinputlisting[float,language=Perl,title={../examples/test012a.pl}]{../examples/test012a.pl}
\clearpage
\subsection[Daten]{Beispiel 12: Messwerte plotten}
Im Beispiel wurde die Kennlinie \(U_{C2}=f(U{B1})\) eines zweistufigen
Transistorverstärkers aufgenommen. Dabei wurden Messwerte entsprechend der
Tabelle gewonnen.\\[1em]
\begin{tabular}{ccc}
\begin{minipage}[m]{0.45\linewidth}
\begin{tabular}{|c|c|}
\hline
\(U_{B1}\) / V&\(U_{C2}\) / V\\
\hline
\hline
&\\[-0.9em]0,350&~0,17\\
\hline
&\\[-0.9em]0,355&~0,21\\
\hline
&\\[-0.9em]0,360&~0,26\\
\hline
&\\[-0.9em]0,365&~0,31\\
\hline
&\\[-0.9em]0,370&~0,38\\
\hline
&\\[-0.9em]0,375&~0,45\\
\hline
&\\[-0.9em]0,380&~0,54\\
\hline
&\\[-0.9em]0,385&~0,65\\
\hline
&\\[-0.9em]0,390&~0,79\\
\hline
&\\[-0.9em]0,395&~0,93\\
\hline
&\\[-0.9em]0,400&~1,10\\
\hline
&\\[-0.9em]0,405&~1,33\\
\hline
&\\[-0.9em]0,410&~1,62\\
\hline
&\\[-0.9em]0,415&~1,90\\
\hline
&\\[-0.9em]0,420&~2,27\\
\hline
&\\[-0.9em]0,425&~2,66\\
\hline
\end{tabular}
\end{minipage}&~\begin{minipage}[m]{0.45\linewidth}
\begin{tabular}{|c|c|}
\hline
\(U_{B1}\) / V&\(U_{C2}\) / V\\
\hline
\hline
&\\[-0.9em]0,430&~3,16\\
\hline
&\\[-0.9em]0,435&~3,65\\
\hline
&\\[-0.9em]0,440&~4,35\\
\hline
&\\[-0.9em]0,445&~5,04\\
\hline
&\\[-0.9em]0,450&~5,86\\
\hline
&\\[-0.9em]0,455&~6,64\\
\hline
&\\[-0.9em]0,460&~7,58\\
\hline
&\\[-0.9em]0,465&~8,69\\
\hline
&\\[-0.9em]0,470&~9,92\\
\hline
&\\[-0.9em]0,475&11,20\\
\hline
&\\[-0.9em]0,480&12,53\\
\hline
&\\[-0.9em]0,485&13,74\\
\hline
&\\[-0.9em]0,490&15,01\\
\hline
&\\[-0.9em]0,495&15,07\\
\hline
&\\[-0.9em]0,500&15,09\\
\hline
&\\[-0.9em]&\\
\hline
\end{tabular}
\end{minipage}
\end{tabular}\\[1em]
Mit der Methode
\textit{set\textunderscore{}xy\textunderscore{}points\textunderscore{}text()\/}
des Plot-Objektes wird als Datenquelle eine Variable angegeben, die den
Text mit den Messwerten enthält. Jeder Messpunkt steht für sich in einer
Zeile. Optional kann -- wie für \(U_{B1}=0,49\,\text{V}\) zu sehen -- der
Anstieg im Punkt als dritter Wert angegeben werden. Dieser wird nur
berücksichtigt, wenn eine Kurve in die Werte hineininterpoliert wird.\\
Mit
\textit{set\textunderscore{}xy\textunderscore{}points\textunderscore{}file()\/}
werden die Messwerte aus einer Datei gelesen.\\
Mit
\textit{set\textunderscore{}xy\textunderscore{}points()\/} wird eine Referenz
auf ein Array angegeben, das wiederum Array-Referenzen für die Daten der
jeweils einzelnen Punkte enthält.

Mit
\textit{set\textunderscore{}dots()\/} wird als Ausgabe eine Punktmenge
anstelle einer Kurve erzeugt.\\
Wird der Befehl
\textit{set\textunderscore{}dots()\/} weggelassen, wird ein Natural Spline
in die Messwerte hineininterpoliert und die sich daraus ergebende Kurve
gezeichnet.
\clearpage
\includepgfimage{../examples/test013a}{Messwerte}
\clearpage
\lstinputlisting[float,language=Perl,title={../examples/test013a.pl}]{../examples/test013a.pl}
\clearpage
\includepgfimage{../examples/test014a}{Messkurve}
\clearpage
\lstinputlisting[float,language=Perl,title={../examples/test014a.pl}]{../examples/test014a.pl}
\clearpage
\subsection[X-Splines]{Beispiel 13: Messwerte plotten mit X-Splines}
\includepgfimage{../examples/test015a}{Messwerte-Plot mit X-Spline}
Da das Modul LaTeX::PGF::Diagram2D keine Glättung zur Verfügung stellt, wirkt
der vorangegangene Plot etwas "`unruhig"'.
Dies kann reduziert werden, indem ein X-Spline-Plot vorgenommen wird.
Ein X-Spline ist eine kontinuierliche Kurve, die durch einen Satz
Kontrollpunkte bestimmt wird. Die Kontrollpunkte bestehen aus einem
\(x\)-Wert, einem \(y\)-Wert und einem Wert \(s\).\\
Für \(-1\leq{}s<0\) erfolgt eine Interpolation der Kurve durch den
angegebenen Punkt, für \(0\leq{}s\leq{}1\) eine Approximation.
Der Wert \(s=0\) erlaubt einen Knick der Kurve am angegebenem Punkt.
Positive Werte \(0<s\leq{}1\) erlauben, dass sich die Kurve leicht vom
angegebenem Punkt hinwegbewegt. Negative Werte veranlassen, dass die Kurve
durch den Punkte gezeichnet wird, \(-1\) sorgt für die geringstmögliche
Krümmung.

Im Beispiel wird für \(U_{\text{B1}}=0,490\,\text{V}\) ein Passieren des
Punktes mit notfalls kleinem Krümmungsradius gefordert, für alle anderen
Messpunkte wird zugeslassen, dass die Kurve minimal versetzt wird.

Zur Kontrolle werden die Messpunkte mit eingezeichnet.
\clearpage
\lstinputlisting[float,language=Perl,title={../examples/test015a.pl}]{../examples/test015a.pl}
\clearpage
\section{Referenz}
\subsection[Diagram]{LaTeX::PGF::Diagram2D}
\subsubsection{Übersicht}
Jedes Objekt der Klasse LaTeX::PGF::Diagram2D repräsentiert ein Diagramm.
\subsubsection{Konstruktor}
\begin{itemize}
\item	\textit{LaTeX::PGF::Diagram2D-\textgreater{}new(width, height)\/}\\
erstellt ein neues Diagramm. Die Breite und Höhe der Zeichenfläche wird
über die Parameter \textit{width\/} und \textit{height\/} vorgegeben,
die Angaben erfolgen in cm.
\end{itemize}
\subsubsection{Setup}
\begin{itemize}
\item	\textit{set\textunderscore{}font\textunderscore{}size(size)\/}\\
gibt die Schriftgröße in Point (pt) vor. Diese Angabe wird genutzt,
um die \(y\)-Koordinate der Skalenteilung für die untere \(x\)-Achse
zu berechnen.
\item	\textit{axis(name)\/}\\
gibt eine Referenz auf ein Objekt der Klasse
"`LaTeX::PGF::Diagram2D::Axis"' zurück. Diese Objektreferenz kann genutzt
werden, um die entsprechende Achse einzurichten.\\
Als Namen sind "`b"', "`l"', "`r"' und "`t"' bzw. "`bottom"', "`left"',
"`right"' und "`top"' erlaubt.
\end{itemize}
\subsubsection{Druckdaten}
\begin{itemize}
\item	\textit{plot(name,name)\/}\\
legt ein neues Objekt der Klasse "`LaTeX::PGF::Diagram2D::Plot"' an und gibt eine
Referenz auf das Objekt zurück. Diese Objektreferenz kann dann genutzt
werden, um den Plot zu konfigurieren.\\
Es sind die gleichen Achsennamen zulässig wie unter "`\textit{axis()}"'
beschrieben.
\item	\textit{copy\textunderscore{}plot(plot)\/}\\
legt ein neues Objekt der Klasse "`LaTeX::PGF::Diagram2D::Plot"' als Kopie
eines bereits vorhandenen an und gibt eine Referenz auf das Objekt zurück.\\
\end{itemize}
\subsubsection{Zusätzliche Graphikobjekte}
\begin{itemize}
\item	\textit{label(name, name, x, y, text [, position])\/}\\
legt ein neues Textobjekt an. Als Argumente müssen die Namen der Achsen
angegeben werden, auf die sich die Koordinaten des Textlabels beziehen,
die Koordinaten des Textlabels und der Text selbst. Optional kann eine
Verankerung (z.\,B. "`left"', "`right"', "`top"', "`bottom"' oder eine
Kombination mittels Komma) angegeben werden.
Standardmäßig wird ein Textlabel horizontal
und vertikal zentriert an den vorgegebenen Koordinaten angebracht.
\item	\textit{polyline(name, name, arrayreference)\/}\\
legt ein Polyline-Objekt an.
Als Argumente müssen die Namen der Achsen
angegeben werden, auf die sich die Koordinaten beziehen.
Das dritte Argument ist eine Referenz auf ein Array, in dem sich für jeden
Punkt der \(x\)- und der \(y\)-Wert befinden.
\end{itemize}
\subsubsection{Ausgabe}
\begin{itemize}
\item	\textit{write(filename)\/}\\
erstellt die Graphikdatei mit dem Diagramm. Hat der Dateiname die Endung
"`.tex"', wird eine vollständige \LaTeX{}-Quelldatei erstellt,
andernfalls eine PGF-Datei, die in eine \LaTeX{}-Quelle eingebunden werden
kann.
\end{itemize}
\clearpage
\subsection[Axis]{LaTeX::PGF::Diagram2D::Axis}
\subsubsection{Übersicht}
Jedes Objekt der Klasse LaTeX::PGF::Diagram2D::Axis repräsentiert eine der vier
möglichen Achsen
\begin{itemize}
\item	b (bottom, untere \(x\)-Achse),
\item	l (left, linke \(y\)-Achse),
\item	r (right, rechte \(y\)-Achse) bzw.
\item	t (top, obere \(x\)-Achse).
\end{itemize}
Die Objekte werden automatisch vom Konstruktor der Klasse
LaTeX::PGF::Diagram2D mit angelegt.

Mit der Diagramm-Methode
\textit{get\textunderscore{}axis(name)\/} erhält man die Referenz auf das
zur jeweiligen Achse gehörige Objekt. Als Namen sind "`b"',
"`l"', "`r"' bzw. "`t"' zulässig, alternativ können
"`bottom"', "`left"', "`right"' bzw. "`top"' genutzt werden.

\subsubsection{Wertebereich}
\begin{itemize}
\item	\textit{set\textunderscore{}linear(min, max)\/}\\
legt eine lineare Skalenteilung für die Achse mit dem angegebenen
Wertebereich fest.
\item	\textit{set\textunderscore{}logarithmic(min, max)}\\
legt eine logarithmische Skalenteilung für die Achse fest und gibt den
Wertebereich vor. Es sind nur positive Werte zulässig.
\end{itemize}

\subsubsection{Skalenteilung}
\begin{itemize}
\item	\textit{set\textunderscore{}grid\textunderscore{}step(step)\/}\\
legt die Schrittweite für die Gitternetzlinien fest (nur bottom und left).
Für logarithmisch geteilte Skalen wird der vorgegebene Wert zum Vervielfachen
des Minimums bzw. des letzten Wertes verwendet.
\item	\textit{set\textunderscore{}tic\textunderscore{}step(step)\/}\\
legt die Schrittweite zum Schreiben der Skalenteilungs-Werte fest.
\end{itemize}

\subsubsection{Achsenbeschriftung, Einheiten}
\begin{itemize}
\item	\textit{set\textunderscore{}label(text)\/}\\
legt den Text für die Achsenbeschriftung fest, diese wird normalerweise
geneigt geschrieben, wenn es sich um ein Formelzeichen handelt.
\item	\textit{set\textunderscore{}unit(text)\/}\\
legt die Einheit fest, diese wird aufrecht geschrieben.
\item	\textit{set\textunderscore{}omit(number)\/}\\
legt fest, wieviele Werte der Skalenteilung weggelassen werden sollen, um
Platz für die Einheit zu schaffen.\\
\emph{Hinweis:} Der Wert "`0"' darf nicht weggelassen werden.
\item	\textit{set\textunderscore{}color(color)\/}\\
legt die Farbe für die Achsenbeschriftung und den Pfeil fest, es können
\LaTeX{}-Farbangaben wie z.\,B. "`blue"' oder "`blue!50!black"'
benutzt werden.\\
Beeinflusst wird nur die Achsenbeschriftung und der Pfeil, die Skalenteilung
bleibt schwarz.
\end{itemize}
\clearpage
\subsubsection{Ränder und Entfernungen}
\figimagelw{}{diagf002}{Abstände}
Für jede Achse gibt die Größe "`to"' die Entfernung vom Diagrammrand
zum rechten, oberen oder unteren Rand der Skalenteilung an.
Die Größe "`lo"' (label offset) gibt die Entfernung vom Diagramm-Rand zum
Mittelpunkt der Achsenbeschriftung an.\\
Die Größe "`bo"' (border) gibt den Abstand zwischen Diagramm-Rand und
dem Rand der Zeichnung an.
\begin{itemize}
\item	\textit{set\textunderscore{}tic\textunderscore{}offset(distance)\/}\hfill{}setzt to.
\item	\textit{set\textunderscore{}label\textunderscore{}offset(distance)\/}\hfill{}setzt lo.
\item	\textit{set\textunderscore{}border(distance)\/}\hfill{}setzt bo.
\end{itemize}
\clearpage
\subsection[Plot]{LaTeX::PGF::Diagram2D::Plot}
\subsubsection{Übersicht}
Jedes Objekt der Klasse "`LaTeX::PGF::Diagram2D::Plot"' repräsentiert einen
in das Diagramm zu druckenden Datensatz.
\subsubsection{Konstruktor}
Der Konstruktor wird automatisch durch die Methoden
\textit{plot()\/} bzw.
\textit{copy\textunderscore{}plot()\/} der Klasse
"`LaTeX::PGF::Diagram2D"' aufgerufen.
\subsubsection{Datenquelle}
\begin{itemize}
\item	\textit{set\textunderscore{}xy\textunderscore{}fct(fct [, derivative ])}\/\\
legt fest, dass zur Gewinnung der Plot-Daten die angegebene Funktion
\textit{fct()\/} aufgerufen wird.\\
Optional kann eine zweite Funktion angegeben werden, die als Ergebnis die
erste Ableitung liefert. Es wird empfohlen, diese Funktion mit anzugeben wenn
möglich.
\item	\textit{set\textunderscore{}parametric\textunderscore{}fct(min, max, xfct, yfct [, xderivative, yderivative ])\/}\\
legt fest, dass parametrisch geplottet werden soll. Für den laufenden
Parameter werden der Start- und der Endwert vorgegeben.
Die weiteren Argumente sind Referenzen zu Funktionen zur Berechnung von
\(x\)- und \(y\)-Wert in Abhängigkeit vom laufenden Parameter und
Funktionen zur Berechnung der ersten Ableitungen.\\
Obwohl die Funktionen zur Berechnung der ersten Ableitung optional sind,
wird empfohlen, diese anzugeben wenn möglich.
\item	\textit{set\textunderscore{}xy\textunderscore{}points(arrayref)\/}\\
gibt ein Array an, das die Koordinaten der Punkte enthält.
Das Array wird als Referenz angegeben, es enthält wiederum Referenzen
zu Arrays mit je einem \(x\)- und einem \(y\)-Wert und optional dem
Anstiegswert des Punktes.
\item	\textit{set\textunderscore{}xy\textunderscore{}points\textunderscore{}text(text)\/}\\
gibt vor, dass die \(x\)- und \(y\)-Werte (und optional der Anstieg) für
die Punkte aus einer Textvariablen gelesen werden.
Jede Textzeile enthält die Daten für einen Punkt.
\item	\textit{set\textunderscore{}xy\textunderscore{}points\textunderscore{}file(filename)\/}\\
gibt vor, dass die Punktkoordinaten aus einer Textdatei eingelesen werden.
Jede Zeile enthält die Daten für einen Punkt, jeweils \(x\)- und
\(y\)-Wert, optional auch den Anstieg der Kurve im Punkt.
\item	\textit{set\textunderscore{}xsplines\textunderscore{}points(arrayref)}\\
gibt ein Array an, dass die Koordinaten der Messpunkte enthält.
Das Array wird als Referenz angegeben, es enthält wiederum Referenzen
zu Arrays mit je einem \(x\)- und einem \(y\)-Wert und optional dem
\(s\)-Parameter des Punktes. Ist \(s\) nicht vorgegeben, wird der Standardwert
\(-1\) verwendet.
\item	\textit{set\textunderscore{}xsplines\textunderscore{}points\textunderscore{}text(text)}\\
gibt vor, dass die \(x\)- und \(y\)-Werte (und optional der \(s\)-Parameter)
für die X-Spline-Punkte aus einer Textvariable gelesen werden.
Jede Textzeile enthält die Daten für einen Punkt.
\item	\textit{set\textunderscore{}xsplines\textunderscore{}points\textunderscore{}file(filename)\/}\\
gibt vor, dass die Punktkoordinaten aus einer Textdatei eingelesen werden.
Jede Zeile enthält die Daten für einen Punkt, jeweils \(x\)- und
\(y\)-Wert, optional einen \(s\)-Wert.
\item	\textit{set\textunderscore{}xsplines\textunderscore{}segments(number)}\\
legt fest, wieviele Bezier-Spline-Segmente pro X-Spline-Segment gezeichnet
werden. Der Standardwert 8 sollte ausreichend sein.
\end{itemize}
\subsubsection{Ausgabe festlegen}
\begin{itemize}
\item	\textit{set\textunderscore{}intervals( number )\/}\\
legt fest, in wieviele Abschnitte der Druckbereich aufgeteilt wird.
Dies wirkt sich nur beim Plotten von Funktionen aus.
\item	\textit{set\textunderscore{}curve()\/}\\
legt fest, dass ein Kurvenzug ausgegeben werden soll.
\item	\textit{set\textunderscore{}lines()\/}\\
legt fest, dass ein Linienzug (polyline) gezeichnet wird.
\item	\textit{set\textunderscore{}dots([ style [, size ]  ])\/}\\
legt fest, dass eine Punktwolke gezeichnet werden soll.
Der optionale Parameter 
\textit{style} gibt vor, wie jeder einzelne Punkt zu zeichnen
ist. Mögliche Werte sind
"`circle"', "`square"', "`diamond"', "`triangle"', "`crosshair"' und
"`pentagon"'.\\
Der optionale zweite Parameter legt das Größenverhältnis des Punktes
im Vergleich zur Linienbreite fest.
\item	\textit{set\textunderscore{}color(name)\/}\\
legt die Farbe für den Plot fest, hier können \LaTeX{}-Farbnamen
benutzt werden.
\end{itemize}
\subsubsection{Druckvorbereitung}
\begin{itemize}
\item	\textit{finish()\/}\\
bereitet den Datensatz zum Drucken vor. Dabei werden die Funktions- und
ggf. die Anstiegswerte berechnet. Die Funktion wird implizit durch die
Funktion \textit{write()\/} des Diagrammes aufgerufen.\\
Ein expliziter Aufruf ist nur erforderlich, wenn Variablen geändert werden
sollen, die als Parameter in die Funktionen eingehen.
\end{itemize}
\clearpage
\subsection{Label}
Jedes Objekt der Klasse "`LaTeX::PGF::Diagram2D::Plot"' repräsentiert ein
Text-Label. Die Methode \textit{label()\/} eines Diagramm-Objektes
erzeugt ein Label-Objekt und gibt die Referenz auf das neue Objekt zurück.

Mit der Methode \textit{set\_color()\/} kann die Farbe des Textlabels eingestellt
werden, es muss eine \LaTeX{}-Farbangabe als Argument verwendet werden.
\clearpage
\subsection{Polyline}
Jedes Object der Klasse "`LaTeX::PGF::Diagram2D::Polyline"' repräsentiert
einen zusätzlichen Linienzug. Die Methode \textit{polyline()\/} eines
Diagramm-Objektes erzeugt ein Polyline-Objekt und gibt die Referenz auf das
neue Objekt zurück.

Mit der Methode \textit{set\_color()\/} kann die Farbe des Textlabels eingestellt
werden, es muss eine \LaTeX{}-Farbangabe als Argument verwendet werden.

Mit der Methode \textit{set\_width()\/} kann die Linienbreite eingestellt
werden, das Argument ist ein Faktor. Die Linienbreite ergibt sich durch
Multiplikation des Faktors mit der Linienbreite von Kurven"=Plots.
\clearpage
\section{Tipps}
\begin{itemize}
\item	\textbf{Geben Sie die erste Ableitung mit an, falls bekannt.}\\
Beim Plotten von Funktionen wird die erste Ableitung
\begin{align*}\frac{\intd Y}{\intd X}&\end{align*}
benötigt, bei parametrischen Plots
\begin{align*}\frac{\intd X}{\intd T}&&\frac{\intd Y}{\intd T}\end{align*}
um daraus die entsprechenden Ableitungen der Verläufe auf dem Papier
\begin{align*}\frac{\intd y}{\intd x}&&\frac{\intd x}{\intd t}&&\frac{\intd y}{\intd t}\end{align*}
zu bilden.\\
Sind nicht alle benötigten Ableitungen gegeben sondern nur eine Funktion
zur Berechnung der Werte bzw. die Werte ohne Ableitungen, versucht das Modul,
ein Natural Spline mit vereinfachten Polynomen durch die Punkte zu legen.
Ist dabei für einen Teil der Punkte ein Anstieg gegeben, wird dieser mit
berücksichtigt.\\
Für jedes Spline-Segment müssen 4 Koeffizienten berechnet werden. Für \(s\)
Splines werden \(4\cdot{}s\) Werte benötigt, es muss also ein lineares
Gleichungssystem mit \(4\cdot{}s\) Gleichungen gelöst werden. Bei einer hohen
Anzahl Segmente kann dies zeitaufwendig werden.\\
Wird jedes Bezier-Segment mit den richtigen Anstiegen berechnet, spiegelt es
den Kurvenverlauf möglicherweise etwas besser wieder als ein Bezier-Segment,
das mit "`geratenen"' Anstiegen produziert wurde.
\item	\textbf{Plotten Sie schwarz/weiß.}\\
In \textsc{din} 461 ist Farbe mit keinem Wort erwähnt.
Technische Dokumente müssen gelegentlich auch unter ungünstigen Umständen
reproduziert werden. Beim Außeneinsatz "`auf der Baustelle"' geht es 
beispielsweise schneller, das nächstbeste Fax zum Kopieren zu verwenden,
als extra wegen einer Farbkopie in das Büro zurückzukehren oder einen
Copy-Shop zu suchen.\\
Um sicherzugehen, dass s/w-Kopien bzw. Kopien auf Faxgeräten zumindest
noch benutzbar sind, sollten Sie Ihre Plots nach Möglichkeit nur in
s/w anfertigen oder zumindest nur dunkle Farben verwenden.\\
Falls Sie mehrere Kurven in ein Diagramm plotten, sollten die Kurven
nicht nur an der Farbe unterscheidbar sein. Verwenden Sie Text-Labels
oder erzeugen Sie eine Legende, um die Unterscheidung
zu gewährleisten.
\end{itemize}
\clearpage
\section{Theorie}
\subsection{Koordinaten}
2D-Plots werden benutzt, um eine Größe \(Y\) in Abhängigkeit von einer
Größe \(X\) darzustellen. Dazu werden auf dem Paper die Koordinaten
\(y\) und \(x\) genutzt.\\
Bei parametrischen Plots sind \(X\) und \(Y\) abhängig von einem Parameter
\(T\), der ein gewisses Intervall durchläuft.

Für die Diagrammachse wird vorgegeben, welcher Wertebereich
\([X_{\text{min}};X_{\text{max}}]\) den Papierkoordinaten im Bereich
\([x_{\text{min}};x_{\text{max}}]\) entsprechen soll.

Die Zuordnung zwischen Größe und Koordinate kann dabei linear
oder logarithmisch sein.

Für lineare Skalenteilungen gilt:
\begin{align*}
x&=x_{\text{min}}+\frac{X-X_{\text{min}}}{X_{\text{max}}-X_{\text{min}}}\cdot{}(x_{\text{max}}-x_{\text{min}})\\
\intertext{Für die erste Ableitung ergibt sich:}
\frac{\intd x}{\intd X}&=\frac{x_{\text{max}}-x_{\text{min}}}{X_{\text{max}}-X_{\text{min}}}\end{align*}
Für logarithmische Skalenteilungen gilt:
\begin{align*}
x&=x_{\text{min}}+(x_{\text{max}}-x_{\text{min}})\frac{\log{\left(\frac{X}{X_{\text{min}}}\right)}}{\log{\left(\frac{X_{\text{max}}}{X_{\text{min}}}\right)}}\\
\intertext{Für die erste Ableitung ergibt sich:}
\frac{\intd x}{\intd X}&=\frac{x_{\text{max}}-x_{\text{min}}}{X\log{\left(\frac{X_{\text{max}}}{X_{\text{min}}}\right)}}
\end{align*}
Wie man sieht, hängt bei logarithmisch geteilten Skalen die Ableitung
vom jeweiligen Wert ab, bei linear geteilten Skalen nicht.

Analoge Beziehungen gelten für die \(y\)-Achse.

Zum Zeichnen von Punkten (Dots) und Linienzügen (Polylines) sind diese
Beziehungen ausreichend.
\clearpage
\subsection{Kurven-Plot mit Bezier-Splines}
\subsubsection{Bezier-Segmente}
Das Plotten von Funktionen erfolgt, indem eine Reihe von Stützstellen
generiert wird, anschließend wird eine Kurve in die Stützstellen hineingelegt.
Die Stützstellen werden so gewählt, dass sie gleiche \(x\)-Abstände
haben (für logarithmisch geteilte Achsen bedeutet dies von Stützstelle
zu Stützstelle eine Vervielfachung des \(X\)-Wertes mit dem gleichen
Faktor).

Die gezeichnete Kurve besteht aus mehreren Segmenten. Modere Graphikformate
(wie PostScript, PDF und das hier genutzte PGF) unterstützen
Bezier-Segmente. Jedes Bezier-Segment wird durch 4 Punkte
\(P_0(x_0;y_0)\) (Anfangspunkt), \(P_0^+(x_0^+;y_0^+)\) (linker Kontrollpunkt),
\(P_1^-(x_1^-;y_1^-)\) (rechter Kontrollpunkt) und
\(P_1(x_1;y_1)\) (Endpunkt) definiert.
Der Kurvenverlauf ist dann eine parametrische
Kurve
\begin{align*}
0&\leq{}t\leq 1\\[0.2em]
x(t)&={(1-t)}^3x_0+3t{(1-t)}^2x_0^++3t^2(1-t)x_1^-+t^3x_1\\[0.2em]
y(t)&={(1-t)}^3y_0+3t{(1-t)}^2y_0^++3t^2(1-t)y_1^-+t^3y_1
\end{align*}
Für die ersten Ableitungen an den Stellen \(t=0\) und \(t=1\) ergibt sich
\begin{align*}
\atpos{\frac{\intd x}{\intd t}}{t=0}&=3x_0^+-3x_0&\atpos{\frac{\intd x}{\intd t}}{t=1}&=3x_1-3x_1^-\\[0.2em]
\atpos{\frac{\intd y}{\intd t}}{t=0}&=3y_0^+-3y_0&\atpos{\frac{\intd y}{\intd t}}{t=1}&=3y_1-3y_1^-\\
\intertext{Sind Anfangspunkt \(P_0\) und Endpunkt \(P_1\) sowie die Anstiege \(\tfrac{\intd x}{\intd t}\) und \(\tfrac{\intd y}{\intd t}\) in beiden Punkten bekannt, können daraus die Kontrollpunkte berechnet werden:}
x_0^+&=x_0+\frac{1}{3}\cdot\atpos{\frac{\intd x}{\intd t}}{t=0}&x_1^-&=x_1-\frac{1}{3}\cdot\atpos{\frac{\intd x}{\intd t}}{t=1}\\[0.2em]
y_0^+&=y_0+\frac{1}{3}\cdot\atpos{\frac{\intd y}{\intd t}}{t=0}&y_1^-&=y_1-\frac{1}{3}\cdot\atpos{\frac{\intd y}{\intd t}}{t=1}
\end{align*}
\clearpage
\subsubsection{Plots y=f(x)}
Bei Plots mit Abhängigkeit \(y=f(x)\) wird für jedes Kurvensegment
mit gegebenem
\[
x_0\qquad{}y_0\qquad\atpos{\frac{\intd y}{\intd x}}{x=x_0}\qquad{}x_1\qquad{}y_1\qquad\atpos{\frac{\intd y}{\intd x}}{x=x_1}
\]
eine Substitution vorgenommen:
\begin{align*}
x(t)&=x_0+(x_1-x_0)\cdot t&t\in[0;1]\\[0.2em]
\frac{\intd x}{\intd t}&=x_1-x_0&\frac{\intd y}{\intd t}&=\frac{\intd y}{\intd x}\cdot\frac{\intd x}{\intd t}
\end{align*}
Ist die Ableitung der Größen \(\tfrac{\intd Y}{\intd X}\) bekannt, so kann
daraus die Ableitung der Kurve auf dem Papier gewonnen werden:
\begin{align*}
\frac{\intd y}{\intd x}&=\frac{\frac{\intd y}{\intd Y}}{\frac{\intd x}{\intd X}}\cdot\frac{\intd Y}{\intd X}
\end{align*}
Ist \(\tfrac{\intd Y}{\intd X}\) nicht bekannt, muss eine Interpolation der
Kurve erfolgen, um die Anstiegswerte zu ermitteln,
siehe Abschnitt~\vref{sec:interpolation}.
\clearpage
\subsubsection{Parametrische Plots}
Für einzelne Kurvensegmente im Bereich von \(T_0\) bis \(T_1\) wird
substituiert:
\begin{align*}
T(t)&=T_0+(T_1-T_0)\cdot t&t&\in[0;1]\\[0.2em]
\frac{\intd T}{\intd t}&=T_1-T_0\end{align*}
Sind die Ableitungen \(\tfrac{\intd X}{\intd T}\) und
\(\tfrac{\intd Y}{\intd T}\) gegeben, können daraus die für die Bezier-Segmente
benötigten Ableitungen \(\tfrac{\intd x}{\intd t}\)
und \(\tfrac{\intd y}{\intd t}\) ermittelt werden:
\begin{align*}
\frac{\intd x}{\intd t}&=\frac{\intd x}{\intd X}\cdot\frac{\intd X}{\intd T}\cdot\frac{\intd T}{\intd t}\\[0.2em]
\frac{\intd y}{\intd t}&=\frac{\intd y}{\intd Y}\cdot\frac{\intd Y}{\intd T}\cdot\frac{\intd T}{\intd t}
\end{align*}
Sind die Ableitungen nicht gegeben, werden \(\tfrac{\intd x}{\intd t}\) und
\(\tfrac{\intd y}{\intd t}\) durch Kurveninterpolation ermittelt,
siehe Abschnitt~\vref{sec:interpolation}.
\clearpage
\subsection{Kurveninterpolation}\label{sec:interpolation}
\subsubsection{Übersicht}
Bei der Kurveninterpolation wird nach Funktionen gesucht, die durch
bestimmte Punkte verlaufen. Durch Differentiation dieser Funktion(en)
kann dann der Anstieg im jeweiligen Punkt ermittelt werden. Weiterhin
können Funktionswerte und Ableitungen auch zwischen den gegebenen Punkten
bestimmt werden.

Dabei gibt es sowohl die Möglichkeit, eine durchgängige Funktion
zu definieren, die alle Punkte trifft als auch die Möglichkeit, die
Interpolationsfunktion stückweise aus verschiedenen Funktionen zusammenzusetzen.

Ein Beispiel für eine durchgängige Kurve wäre ein Polynom höheren Grades.
Sind \(N\) Punkte gegeben, kann daraus ein Gleichungssystem mit \(N\)
Gleichungen erstellt und gelöst werden, um z.\,B. die \(N\) Koeffizienten
eines Polynomes vom Grad \(N-1\) zu bestimmen.\\
Dieses Verfahren wird aber praktisch nicht verwendet, bzw. so selten, dass
mir nichts davon bekannt ist. Der Nachteil eines Polynomes vom Grad
\(N-1\) ist, dass es bis zu \(N-2\) Extremstellen aufweist. Liegen einige
davon zufälligerweise zwischen den gegeben Punkten, erhalten wir eine stark
oszillierende Kurve.

Daher verwendet man meist andere Verfahren, wie z.\,B. Natural Splines
oder X-Splines, bei denen der Einfluss einzelner Punkte überwiegend lokal ist.
\subsubsection{Natural Splines}
Für \(N\) gegebene Punkte mit streng monoton steigenden \(x\)- bzw. \(t\)-Werten
werden \(N-1\) Segmente gebildet. Jedes Segment \(i\) zwischen den
\(x\)-Werten \(x_i\) und \(x_{i+1}\)
wird durch ein kubisches
Polynom
\begin{align*}
y&=m_i{(x-x_i)}^3+n{(x-x_i)}^2+p_i(x-x_i)+q_i\\
\intertext{bzw.}
y&=a_ix^3+b_ix^2+c_ix+d_i
\end{align*}
beschrieben. Beide Schreibweisen sind gleichwertig,
(\(m_i\), \(n_i\), \(p_i\), \(q_i\)) und (\(a_i\), \(b_i\), \(c_i\), \(d_i\))
können ineinander umgerechnet werden. Wir verwenden die zweite Schreibweise,
da sie einige Berechnungen einfacher ermöglicht.
\clearpage
Für \(N-1\) Kurvensegmente haben wir \(4(N-1)\) Koeffizienten zu finden,
somit müssen wir \(4(N-1)\) Gleichungen aufstellen:
\begin{itemize}
\item	N-1 Gleichungen (nummeriert mit 0\ldots{}N-2) für den Funktionswert
am linken Rand eines jeden Segmentes (\(0\leq i\leq N-1\)):
\begin{align*}y_i&=a_ix_i^3+b_ix_i^2+c_ix_i+d_i\end{align*}
\item	N-1 Gleichungen (nummeriert mit N-1\ldots{}2N-3) für den Funktionswert
am rechten Rand eines jeden Splines (\(0\leq i\leq N-1\)):
\begin{align*}y_{i+1}&=a_ix_{i+1}^3+b_ix_{i+1}^2+c_ix_{i+1}+d_i\end{align*}
\item	N-2 Gleichungen (nummeriert mit 2N-2\ldots{}3N-5), die dafür sorgen,
dass an inneren Punkten die erste Ableitung des linken Segmentes gleich
der ersten Ableitung des rechten Segmentes ist (\(0\leq i\leq N-2\)):
\begin{align*}3a_ix_{i+1}^2+2b_ix_{i+1}+c_i&=3a_{i+1}x_{i+1}^2+2b_{i+1}x_{i+1}+c_{i+1}\end{align*}
\item	N-2 Gleichungen (nummeriert mit 3N-4\ldots{}4N-7), die dafür sorgen,
dass an den inneren Punkten die zweite Ableitung des linken Segmentes
gleich der zweiten Ableitung des rechten Segmentes ist
(\(0\leq i\leq N-2\)):
\begin{align*}6a_ix_{i+1}+2b_i&=6a_{i+1}x_{i+1}+2b_{i+1}\end{align*}
\item	2 Gleichungen (nummeriert mit 4N-6 und 4N-5), die festlegen,
dass die zweite Ableitung an den äußeren Punkten 0 ist:
\begin{align*}
6a_0x_0+2b_0&=0\\[0.2em]
6a_{N-2}x_{N-1}+2b_{N-2}&=0\end{align*}
\end{itemize}
Die gesuchten Variablen in diesem Gleichungssystem sind
die (\(a_i\), \(b_i\), \(c_i\) und \(d_i\)).

Sind für einzelne Punkte \(P_i\) die ersten Ableitungen gegeben,
müssen einzelne Gleichungen ersetzt werden.
Handelt es sich um einen inneren Punkt, wird die Gleichung
\(3N-4+i-1\),
die bisher die zweite Ableitung des linken und rechten Polynomes gleichsetzte,
ersetzt durch die Gleichung
\begin{align*}3a_ix_i^2+2b_ix_i+c_i&=y'(x_i)\end{align*}
Handelt es sich um den Anfangspunkt, wird die Gleichung \(4N-6\), die
bisher die zweite Ableitung am Anfangspunkt auf 0 setzte, ersetzt durch
\begin{align*}3a_0x_0^2+2b_0x_0+c_0&=y'(x_0)\end{align*}
Handelt es sich um den letzten Punkt, wird die Gleichung \(4N-5\), welche
bisher die zweite Ableitung in diesem Punkt auf 0 setzte, ersetzt durch
\begin{align*}3a_{N-2}x_{N-1}^2+2b_{N-2}x_{N-1}+c_{N-2}&=y'(x_{N-1})\end{align*}

Das Modul "`LaTeX::PGF::Diagram2D"' verwendet das Modul "`PDL"', um eine numerische
Lösung des Gleichungssystems zu finden.
\clearpage
\subsubsection{X-Splines}
Cross-Splines (X-Splines) wurden von Carole Blanc und Christophe Schlick
auf der \textsc{siggraph}'95 vorgestellt, der Beitrag ist in den
Proceedings abgedruckt und online als "`X-Splines: A Spline Model
Designed for the End-User"' veröffentlicht.\\
Hier werden X-Splines stark vereinfacht beschrieben, soweit es für das
Verständnis des Modules "`LaTeX::PGF::Diagram2D"' erforderlich ist. Für eine
vollständige Beschreibung der X-Splines sei auf die Konferenz-Proceedings
verwiesen.

Ein X-Spline wird als ein Satz von Kontrollpunkten angegeben,
die einem laufenden Parameter \(t\) an den diskreten Stellen
\(t_0\), \(t_1\)\ldots{}\(t_{N-1}\) jeweils Werte \(x_i\), \(y_i\) und
\(s_i\) zuweisen. Optional ist auch ein \(z_i\) möglich.
Für das Diagramm-Modul wird davon ausgegangen, dass die Stützstellen
äquidistant mit dem Abstand 1 sind, d.h.
\begin{align*}\forall i(0\leq i<N-1):&&&t_{i+1}-t_{i}=1\end{align*}
Jeder Punkt weist eine linksseitige und eine rechtsseitige Gewichtsfunktion
auf, die sich maximal bis zum übernächsten Punkt erstrecken kann.

In einem beliebigen Segment zwischen den Stützstellen \(t_i\) und \(t_{i+1}\)
werden \(x\)-Werte mit der Formel
\begin{align*}x(t)&=\frac{x_{i-1}F_{i-1}^+(t)+x_iF_i^+(t)+x_{i+1}F_{i+1}^-(t)+x_{i+2}F_{i+2}^-(t)}{ F_{i-1}^+(t)+F_i^+(t)+F_{i+1}^-(t)+F_{i+2}^-(t)}\end{align*}
gebildet, dabei bezeichnet \(F_j^+(t)\) die rechtsseitige Gewichtsfunktion
des Punktes \(P_j\) und \(F_k^-(t)\) die linksseitige Gewichtsfunktion
des Punktes \(P_k\).

Als Gewichtsfunktionen kommen folgende zwei Funktionen zum Einsatz:
\begin{align*}
\intertext{Für die Approximation:}
f(u)&=\begin{cases}
u^3\big(10-p+(2p-15)u+(6-p)u^2\big)&\text{für~}0\leq{}u\leq{}1\\
0&\text{ansonsten}\\
\end{cases}\\
\intertext{Für die Interpolation:}
e(u)&=\begin{cases}
qu+2qu^2-2qu^4-qu^5&\text{für~}-1\leq{}u<0\\
qu+2qu^2+(10-12q-p)u^3+(2p+14q-15)u^4+(6-5q-p)u^5&\text{für~}0\leq{}u\leq{}1\\
0&\text{ansonsten}\\
\end{cases}
\end{align*}
\clearpage
Der Parameter \(s_i\) bestimmt Typ und Parameter der rechtsseitigen
Gewichtsfunktion des linken Nacharn und der linksseitigen Kontrollfunktion
des rechten Nachbarn.\\
Für \(0\leq{}s_i\leq{}1\) gilt:
\begin{align*}
T_{i-1}^+&=t_i+s_i&p_{i-1}^+&=2{(T_{i-1}^+-t_{i-1})}^2\\[0.2em]
F_{i-1}^+(t)&=f\left(\frac{T_{i-1}^+-t}{T_{i-1}^+-t_i}\right)\\[0.2em]
T_{i+1}^-&=t_i-s_i&p_{i+1}^+&=2{(t_{i+1}-T_{i+1}^-)}^2\\[0.2em]
F_{i+1}^-(t)&=f\left(\frac{t-T_{i+1}^-}{t_{i+1}-T_{i+1}^-}\right)
\end{align*}
Für \(-1\leq{}s_i<0\) gilt:
\begin{align*}
p_{i-1}^+&=2&q_{i-1}^+&=-\tfrac{1}{2}s_i\\[0.2em]
F_{i-1}^+(t)&=e(t_i-t)\\[0.2em]
p_{i+1}^-&=2&q_{i+1}^-&=-\tfrac{1}{2}s_i\\[0.2em]
F_{i+1}^-(t)&=e(t-t_i)
\end{align*}
\clearpage
\subsection{Beispiele}
\subsubsection{Punktmenge}
\includepgfimage{../examples/test016a}{Punktmenge}
In diesem Abschnitt werden einige Beispiele für Kurveninterpolation
und -approximation gezeigt. Dabei soll die angegebene Punktmenge verwendet
werden.
\clearpage
\subsubsection{Polynom hoher Ordnung}
Im ersten Versuch wollen wir ein Polynom 8. Ordnung durch die 9 Messpunkte
interpolieren.
\begin{align*}
y&=ax^8+bx^7+cx^6+dx^5+ex^4+fx^3+gx^2+hx+i
\end{align*}
Das Gleichungssystem
\begin{align*}
\left(\begin{array}{ccccccccc}
1^8&1^7&1^6&1^5&1^4&1^3&1^2&1&1\\
2^8&2^7&2^6&2^5&2^4&2^3&2^2&2&1\\
3^8&3^7&3^6&3^5&3^4&3^3&3^2&3&1\\
4^8&4^7&4^6&4^5&4^4&4^3&4^2&4&1\\
5^8&5^7&5^6&5^5&5^4&5^3&5^2&5&1\\
6^8&6^7&6^6&6^5&6^4&6^3&6^2&6&1\\
7^8&7^7&7^6&7^5&7^4&7^3&7^2&7&1\\
8^8&8^7&8^6&8^5&8^4&8^3&8^2&8&1\\
9^8&9^7&9^6&9^5&9^4&9^3&9^2&9&1\\
\end{array}\right)\cdot\left(\begin{array}{c}a\\b\\c\\d\\e\\f\\g\\h\\i\end{array}\right)&=\left(\begin{array}{c}2\\2\\2\\2\\4\\6\\6\\6\\6\end{array}\right)
\end{align*}
wird numerisch gelöst, dies ergibt:
\begin{align*}
a&=1,7647\cdot{}{10}^{-16}&b&=-1,9841\cdot{}{10}^{-3}&c&=6,9444\cdot{}{10}^{-2}&d&=-9,8056\cdot{}{10}^{-1}\\[0.2em]
e&=7,1528&f&=-28,722&g&=62,778&h&=-68,295\\[0.2em]
i&=30
\end{align*}
Abb.~\vref{fig:../examples/test017a} zeigt das Polynom. Zu sehen sind starke Oszillationen
und ein sehr starker Anstieg an den Enden der Punktmenge.
\clearpage
\includepgfimage{../examples/test017a}{Interpolation mit Polynom 8. Ordnung}
\clearpage
\subsubsection{Natural Spline}
Im nächsten Beispiel legen wir einen Natural Spline durch die Punktmenge.
\lstinputlisting[language=Perl,title={../examples/test018a.pl}]{../examples/test018a.pl}

Im Ergebnis (Abb.~\vref{fig:../examples/test018a}) sind noch schwache Oszillationen
zu sehen.
\clearpage
\includepgfimage{../examples/test018a}{Natural Spline}
\clearpage
Nun geben wir für die Punkte (4;2) und (6;6) vor, dass dort ein Anstieg
0 eingestellt werden soll, da die Kurve ja dort "`von links kommt"' bzw.
"`nach rechts läuft"'.
\lstinputlisting[language=Perl,title={../examples/test019a.pl}]{../examples/test019a.pl}
Im Ergebnis (Abb.~\vref{fig:../examples/test019a}) ist zu sehen, dass die Oszillationen
beseitigt sind.
\clearpage
\includepgfimage{../examples/test019a}{Natural Spline mit Vorgaben für den Anstieg}
\clearpage
\subsubsection{X-Splines}
Im nächsten Beispiel nehmen wir eine Kurveninterpolation mit X-Splines
vor. Da wir keinen \(s\)-Parameter angegeben haben, wird der
Standardwert -1 verwendet, der eine Interpolation bewirkt.
\lstinputlisting[language=Perl,title={../examples/test020a.pl}]{../examples/test020a.pl}
Das Ergebnis ist in Abb.~\vref{fig:../examples/test020a} zu sehen.
\clearpage
\includepgfimage{../examples/test020a}{X-Spline mit Interpolation}
\clearpage
Nun setzen wir manuell den \(s\)-Parameter für die Punkte (4;2) und (6;6)
auf \(-\tfrac{1}{2}\). Dies bewirkt, dass an den Punkten eine stärkere
Krümmung zugelassen wird. Damit werden die Oszillationen verringert.
\lstinputlisting[language=Perl,title={../examples/test021a.pl}]{../examples/test021a.pl}
\includepgfimage{../examples/test021a}{X-Spline, Interpolation mit starker Krümmung}
\clearpage
Um eine Approximation durchzuführen, verwenden wir als Standardwert
für \(s\) 1,0, diesen Wert geben wir in der Funktion
\textit{set\textunderscore{}xsplines\textunderscore{}points\textunderscore{}text()\/}
als zweites Argument an. Der Standardwert wird für alle Zeilen verwendet,
die nicht explizit einen \(s\)-Wert aufweisen.
\lstinputlisting[language=Perl,title={../examples/test022a.pl}]{../examples/test022a.pl}
Die Approximationskurve ist in Abb.~\vref{fig:../examples/test022a} zu sehen.
\clearpage
\includepgfimage{../examples/test022a}{X-Spline mit Approximation}
\clearpage
Wir ändern nun den \(s\)-Wert der Punkte (4;2) und (6;6) auf 0,5.
Damit wird die Kurve näher zu diesen Punkten herangezogen.
\lstinputlisting[language=Perl,title={../examples/test023a.pl}]{../examples/test023a.pl}
\clearpage
\includepgfimage{../examples/test023a}{X-Spline, Approxmiation mit Heranziehen der Kurve}
\clearpage
In der letzten Kurve nehmen wir zahlreiche Änderungen an den \(s\)-Werten
vor, zum zu zeigen, dass sowohl Approximations- als auch Interpolationsbereiche
gleichzeitig möglich sind.
\lstinputlisting[language=Perl,title={../examples/test024a.pl}]{../examples/test024a.pl}
\clearpage
\includepgfimage{../examples/test024a}{X-Spline, Approximation und Interpolation}
\clearpage
\end{document}

